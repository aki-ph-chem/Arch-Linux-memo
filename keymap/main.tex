\documentclass[dvipdfmx]{jsarticle}

\begin{document}
\title{Linuxのキーマップ}
\author{aki}
\thispagestyle{empty}
\maketitle

\section{はじめに}
自分のキーボードRK61は/を入力するのにFn + /を押さなければならず面倒であった。
そこでxmodmapを用いて/のみで入力できるように変更した。

\section{keycoceを調べる}

まず、キーのkeycodeを調べる。ターミナルで

\begin{quote}
    \begin{verbatim}
$ xev
    \end{verbatim}
\end{quote}

として、特定のキーを押し表示されるkeycodeを読み取る。デフォルトでは読みにくいのでgrepを使って以下のようにするとわかりやすい。

\begin{quote}
    \begin{verbatim}
$ xev | grep keycode
    \end{verbatim}
\end{quote}

\section{キーを入れ替える}

先程調べたkeycodeを用いて以下のように.bashrcに書き込む。

\begin{quote}
    \begin{verbatim}
# Up <-> Slash
xmodmap -e 'keycode 111 = slash '
xmodmap -e 'keycode 61 = Up'
    \end{verbatim}
\end{quote}


\end{document}
