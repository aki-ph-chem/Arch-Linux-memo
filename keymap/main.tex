\documentclass[dvipdfmx]{jsarticle}

\begin{document}
\title{Linuxのキーマップ}
\author{aki}
\thispagestyle{empty}
\maketitle

\section{はじめに}
自分のキーボードRK61は/を入力するのにFn + /を押さなければならず面倒であった。
そこでxmodmapを用いて/のみで入力できるように変更した。

\section{keycoceを調べる}

まず、キーのkeycodeを調べる。ターミナルで

\begin{quote}
    \begin{verbatim}
$ xev
    \end{verbatim}
\end{quote}

として、特定のキーを押し表示されるkeycodeを読み取る。デフォルトでは読みにくいのでgrepを使って以下のようにするとわかりやすい。

\begin{quote}
    \begin{verbatim}
$ xev | grep keycode
    \end{verbatim}
\end{quote}

\section{キーを入れ替える}

先程調べたkeycodeを用いて以下のように.bashrcに書き込む。

\begin{quote}
    \begin{verbatim}
# Up <-> Slash
xmodmap -e 'keycode 111 = slash '
xmodmap -e 'keycode 61 = Up'
    \end{verbatim}
\end{quote}

\section{本格版}

.bashrcに直接書き込むより、.Xmodmapにkeymapの設定を書き込んで、.bashrcにxmod ~/.Xmodmapと書き込んだほうがよい。

\subsection{追記:自分の使っているRK61について}

keymapをいじって/?と矢印キーを入れ替えていたが、そもそも自分のRK61ではFn + Enterを押すと/?メインか矢印キーメインかを入れ替えられるのであんまいじらなくても良かった。


\section{おまけ: 尊師スタイル}

gccやgdbなどのコントリビューターとして有名で、GNUの創設者であるリチャードストルマンはラップトップpcにHHKBを乗っけて使っていた。
かれは尊師として尊敬されているため、このスタイルを尊師スタイルという。


まず、現在認識されているデバイス一覧を表示する。
\begin{quote}
    \begin{verbatim}
$ xinput
    \end{verbatim}
\end{quote}

内蔵キーボードは"AT Translated Set 2 keyboard"という名前のデバイスとして認識されている。
なのでこれを無効化する。

\begin{quote}
    \begin{verbatim}
$ xinput -disable "AT Translated Set 2 keyboard" 
    \end{verbatim}
\end{quote}

やっぱり内蔵キーボードを使いたい場合では、もう一回有効化する。

\begin{quote}
    \begin{verbatim}
$ xinput -enable "AT Translated Set 2 keyboard" 
    \end{verbatim}
\end{quote}

入力内容が長くてめんどくさいので.bashrcにaliasとして書き込むと便利である。

\begin{quote}
    \begin{verbatim}
# 内蔵キーボードの無効化
alias diskey='xinput -disable "AT Translated Set 2 keyboard"' 
# 内蔵キーボードの再有効化
alias enkey='xinput -enable "AT Translated Set 2 keyboard"' 
    \end{verbatim}
\end{quote}


\end{document}
