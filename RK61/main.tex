\documentclass[dvipdfmx]{jsarticle}

\begin{document}
\title{RK61のキー事情}
\author{aki}
\thispagestyle{empty}
\maketitle

\section{はじめに}

謎に人気な60のUS配列のキーボードであるRK61を現在愛用中であるが、
中々クセがあって使い方を忘れることがあるのでメモを備忘録を書いておく。


bluetoothと2.4GHzとusb接続の三タイプが選べるが、bluetoothはめんどくさいので今の所使わっていない。

\section{矢印キー事情}

RK61では矢印キーと/が同じキーになっているがFn + Enterで矢印キーメインか、/キーメインかを切り替えることができる。

\section{Fキー事情}

\begin{enumerate}
    \item Fn + 数字でF1 ~ F12が打てる。
    \item Fn + Ctrlでメディアキー Fn + Ctrl2回で元に戻る。
\end{enumerate}

\section{副キー事情}

デフォルトでは主キーが出るが、Fn + Ctrl で主機能と複機能を切り替える。  
もう一回Fn + Ctrlを押すとY,U,H,J,K,N,M,<の機能が元に戻る。
もう一回Fn + Ctrlを押すと元に戻る


\end{document}
