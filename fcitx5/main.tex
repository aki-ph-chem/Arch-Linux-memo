\documentclass[dvipdfmx]{jsarticle}

\begin{document}
\title{fcitxからfcitx5への移行}
\author{aki}
\thispagestyle{empty}
\maketitle

\section{はじめに}
Arch Linuxのfcitxは現在メンテナンスモードでfcitx5への移行が推奨されている。
以下ではfcitx5への移行に必要な作業を示す。

\section{fcitxの削除}

まずインストールされているfcitx関連のパッケージを確認する。

\begin{quote}
    \begin{verbatim}
$ pacmam -Q | grep fcitx
    \end{verbatim}
\end{quote}

自分の環境ではfcitx-mozcを使用しているので以下を削除する。

\begin{quote}
    \begin{verbatim}
$ sudo pacman -Rs fcitx-mozc fcitx-configtool
    \end{verbatim}
\end{quote}

\section{fcitx5のインストール}

これまででfcitx5をインストールする準備ができたのでいよいよインストールしていく。

\begin{quote}
    \begin{verbatim}
$ sudo pacman -S fcitx5-im fcitx5-mozc
    \end{verbatim}
\end{quote}

現在自分が用いているxfce4ではインストールした時点で自動でfcitx5が起動するようになっていた。


再起動後にfcitx5-configを開いてmozcを追加する。


\end{document}
